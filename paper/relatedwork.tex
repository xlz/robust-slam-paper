\section{Related Work}

Graph SLAM has multiple highly efficient optimization solutions.
iSAM\cite{isam} converts the graph SLAM maximum likelihood estimate into a
non-linear least squares optimization problem, then incrementally solves it by
numerical methods, obtaining real-time performance and smoother accuracy. These
optimization techniques show the effectiveness of the factor graph formulation
of the SLAM problem, and we base our formulation in similar forms.

A known solution to moving objects in the environment is combining SLAM with
object detection and tracking. \cite{wang2003online} proposes a Bayesian
framework to solve the SLAM together with object motion modeling by
sophisticated object detection and tracking and data association algorithms.
This approach is suitable for dense data such as laser point-clouds, but less
so for sparse data such as landmarks or features keypoints generated by visual
sensors which is usually less than sufficient to achieve the same level of
accuracy.  Moreover, object detection and tracking is used as a preprocessing
front-end to filter out moving objects. If the goal is just to estimate the
robot trajectory and the still part of the map for future localization, the
work of maintaining motion models of individual moving objects will not be
essential to the SLAM problem. 

Robust SLAM techniques have been proposed to solve front-end outlier problems
without relying on pre-filtering. Some use robust objective functions or robust
representation of graph factors, such as Max-Mixture\cite{mm} which represents
factors not as a single spatial Gaussian distribution, but as a mixture of
Gaussians. This kind of approaches still assume sources of errors being mostly
perceptual aliasing in wrong loop closures, without regard to environmental movement.
Given dynamic environments they will not be able to detect and handle the
movement of landmarks.

The other approach to handling front-end outlier and dynamism is incorporating
the problem of determining mobility as part of the back-end graph optimization
framework.  \cite{haehnel03iros} and \cite{rogers2010slam} extend the graphical
model with a latent indicator variable to each landmark indicating whether it
is mobile and use the EM algorithm to iteratively infer those additional
variables in the graphical model and estimate the optimal solution in SLAM
adjusted with the indicators. The switchable constraints\cite{Switchable12}
approach allows the optimizer to naturally change the topological structure of
the problem during the optimization itself using switch variables as a
multiplicative scaling factor on the information matrix associated with that
constraint. However, these EM based algorithms suffer from the limitation of
the EM algorithm in which they do not have the guarantee of convergence in
specified number of iterations, making them unsuitable for real-time
applications where the optimization must complete before the next measurement
arrives.
