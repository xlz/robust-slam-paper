\section{RELATED WORK}

Graph SLAM has multiple highly efficient optimization solutions.
iSAM \cite{isam} converts the graph SLAM maximum likelihood estimate into a
non-linear least squares optimization problem.  The factor graph is incrementally solved by numerical methods, obtaining real-time performance and Bayesian smoothing accuracy. These optimization techniques show the effectiveness of the factor graph formulation of the SLAM problem, and we base our formulation on similar formulations.

%resolved\cj{let's also look at Wolf and Sukhatme, http://robotics.usc.edu/publications/media/uploads/pubs/392.pdf xlz: fixed}

A known solution to SLAM in dynamic environments is to maintain two occupancy
maps modeling the dynamic and static parts of the environment \cite{wolf05}. By
differentiating dynamic and static parts of the environment with different
representation, this method is capable of mapping and localization in dynamic
environments over time. An alternative approach of dealing with
moving objects in dynamic environments is combining SLAM with
object detection and tracking. Wang et al. \cite{wang2003online} proposed a Bayesian
framework to solve the SLAM together with object motion modeling by
sophisticated object detection and tracking and data association algorithms.
In this approach, object detection and tracking is used as a preprocessing
front-end to filter out moving objects. 
%resolved\cj{Is the next sentence necessary? xlz: no.}
%If the goal is just to estimate the
%robot trajectory and the still part of the map for future localization, there
%are possible solutions without maintaining motion models of individual moving objects.

%resolved\cj{comment: use ' \~ ' to bind citations to text such that they are on the same line. xlz: will manually fix it when seeing more}
Robust SLAM techniques have been proposed to solve front-end outlier problems
without relying on pre-filtering. Some use robust objective functions or robust
representation of observations. Dynamic Covariance Scaling~\cite{DCS} adds a
robust kernel factor to regularize the Mahalanobis errors in the Gaussian
distributions of landmark observations.  Max-Mixture~\cite{mm} enhances factor
potentials with a clever representation for mixtures of Gaussians in place of a unimodal Gaussian distribution. This kind of approaches still assume sources of errors being mostly perceptual aliasing in wrong loop closures, without regard to
environmental movement.  Unless modeled explicitly in factor potentials, these methods will have difficulty in handling the movement of landmarks.

%Given dynamic environments they will have difficulty in detecting and handling the movement of landmarks.

%The other approach to handling 
Front-end outliers and dynamic elements can also be addressed through identifying mobility as part of the back-end graph
optimization framework.  Haehnel et al. \cite{haehnel03iros} and Rogers et al. \cite{rogers2010slam} extend graphical model formulations with a latent indicator variable to infer whether a landmark is mobile. EM algorithms are used to iteratively infer these
latent landmark mobility variables in the graphical model and estimate the optimal SLAM solution.% in SLAM augmented with the indicators. 
The switchable
constraints \cite{Switchable12} approach allows the optimizer to naturally
change the topological structure of the problem during the optimization itself
using switch variables as a multiplicative scaling factor on the information
matrix associated with that constraint. However, these EM based algorithms lack
the robustness provided by previous techniques. %resolved\cj{Can we discuss why this next statement is true? xlz: It should be true by definition. This looks clarified? cjenkins: yes, thanks} 
Further, an observation-based
indicator only models the observation it associates with and will not characterize the
mobility of landmarks which associate with multiple observations.
