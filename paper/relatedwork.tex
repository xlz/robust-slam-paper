\section{Related Work}

Graph SLAM has multiple highly efficient optimization solutions.
iSAM \cite{isam} converts the graph SLAM maximum likelihood estimate into a
non-linear least squares optimization problem.  The factor graph is incrementally solved by numerical methods, obtaining real-time performance and Bayesian smoothing accuracy. These optimization techniques show the effectiveness of the factor graph formulation of the SLAM problem, and we base our formulation on similar formulations.

\cj{let's also look at Wolf and Sukhatme, http://robotics.usc.edu/publications/media/uploads/pubs/392.pdf}

A known solution to moving objects in the environment is combining SLAM with
object detection and tracking. \cite{wang2003online} proposes a Bayesian
framework to solve the SLAM together with object motion modeling by
sophisticated object detection and tracking and data association algorithms.
This approach is suitable for dense data such as laser point-clouds, but less
so for sparse data such as landmarks or features keypoints generated by visual
sensors which is usually less than sufficient to achieve the same level of
accuracy.  Moreover, object detection and tracking is used as a preprocessing
front-end to filter out moving objects. If the goal is just to estimate the
robot trajectory and the still part of the map for future localization, the
work of maintaining motion models of individual moving objects will not be
essential to the SLAM problem. 

Robust SLAM techniques have been proposed to solve front-end outlier problems
without relying on pre-filtering. Some use robust objective functions or robust
representation of observations. Dynamic Covariance Scaling\cite{DCS} adds a
robust kernel factor to regularize the Mahalanobis errors in the Gaussian
distributions of landmark observations.  Max-Mixture\cite{mm} enhances factor
representation with a mixture of Gaussians instead of a single spatial Gaussian
distribution. This kind of approaches still assume sources of errors being
mostly perceptual aliasing in wrong loop closures, without regard to
environmental movement.  Given dynamic environments they will have difficulty
in detecting and handling the movement of landmarks.

The other approach to handling front-end outlier and dynamic elements is
incorporating the problem of identifying mobility as part of the back-end graph
optimization framework.  \cite{haehnel03iros} and \cite{rogers2010slam} extend
the graphical model with a latent indicator variable for each landmark to
indicate whether it is mobile and use EM algorithms to iteratively infer those
additional latent variables in the graphical model and estimate the optimal
solution in SLAM augmented with the indicators. The switchable
constraints \cite{Switchable12} approach allows the optimizer to naturally
change the topological structure of the problem during the optimization itself
using switch variables as a multiplicative scaling factor on the information
matrix associated with that constraint. However, these EM based algorithms lack
the robustness provided by previous techniques.  And observation based
indicators will not be able to characterize the mobility of each landmark which
associates with multiple observations.
