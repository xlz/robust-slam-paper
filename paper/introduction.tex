\section{Introduction}
Simultaneous localization and mapping (SLAM) is a central problem for
autonomous robot navigation. \cj{*run on sentence, split* Graph SLAM formulate it as an inference problem
on a factor graph, where landmark locations and robot poses are the hidden
variables nodes to be mapped and localized, and spatial measurements are the
observed factor nodes as constraints between variable nodes.} The goal of
the inference problem is to obtain the maximum likelihood estimate of the
joint probability of the graph, which becomes the geometrically consistent
estimate of the robot's trajectory and the map. This maximum likelihood estimate on factor graphs can be solved by belief propagation \cj{perhaps cite Weiss et al.}, or more recently, by numerical methods after converting into a nonlinear squares optimization \cj{cite iSAM?}.  

Problems arise in SLAM loop closures when factors incorrectly link unrelated variable nodes.  Such false positives effectively create wormholes between spatially distant locations and, thus, distort the map geometry. In pose graphs, loop closures specify spatial localities between arbitrary poses, but often wrongly connect randomly poses because of bad decisions from the front-end.  In landmark-based graphs, the data association process can also easily obtain wrong visual feature correspondences, connecting landmarks to the wrong poses. This necessitates robust SLAM.

Another issue beyond robust SLAM is dynamic elements in the environment.
Typical SLAM techniques in the literature are designed for unmanned navigation
in uninhabited areas, thus mostly assume static environments with stationary
landmarks or loop closures. However, recent human-robot interaction research
has seen more applications of navigation in populated, crowded, or social
environments where people and furniture moving around is the major
characteristics. If the landmarks are moving, the current localization is either
kidnapped by the movement, or resulting in distorted maps.

In this paper we devise new graph representations and algorithms to
address the issue of front-end outliers and the issue of environmental
dynamic elements within the factor graph framework. We evaluate existing
approaches of robust SLAM and conduct experiments on real world datasets to
validate the performance of our approach.
