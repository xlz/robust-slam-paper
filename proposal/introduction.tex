\section{Introduction}
Simultaneous localization and mapping (SLAM) is the central problem of
autonomous robot navigation, also relates to 3-D reconstruction and
augmented reality. Graph SLAM formulate it as an inference problem on a factor
graph, where landmark locations and robot poses are the hidden variables nodes
to be mapped and localized, and spatial measurements are the observed factor
nodes as constraints between variable nodes. Then the goal of the inference
problem is to obtain the maximum likelihood estimate of the joint probability
of the graph, which becomes the geometrically consistent estimate of robot
trajectory and the map. This maximum likelihood estimate on factor graphs can
be solved by belief propagation, or more recently, by numerical methods after
converting into a non-linear square optimization.

Problems arise when factors incorrectly link unrelated variable nodes,
effectively creating wormholes between spatially distant locations thus
distorting the map geometry. Various circumstances can result in wrong
factors. For pose only graph, the robot continously acquires odometric
measurements, which represent factors between the previous and next robot
poses, and occasionally acquires loop closures, specifying spatial localities
between arbitrary poses. Loop closures are provided by the front-end using
algorithms based on certain similarities and inevitably make mistakes.  For
landmark-based graph, the data association process can also easily obtain
wrong visual feature correspondances, that is, connecting landmarks to the
wrong poses. Hence, accurate SLAM necessitates robust handling of front-end
outliers.

Another issue beyond robust SLAM is dynamism in the environment. Typical SLAM
approaches in the literature are designed for unmanned navigation in
uninhabited areas, and thus mostly assume a static environment with stationary
landmarks or loop closures. However, recent human-robot interaction research
has seen more applications of navigation in populated, crowded, or social
environments where people and furniture moving around is the major
characterstics. If the landmarks are moving, then by the current methods,
localization is either carried away by the movement, resulting in the kidnapped
robot problem, or more inconsistency happens and the map gets distorted. 

In this paper we plan to devise new graph representations and algorithms to
address the issue of front-end outliers and the issue of environmental dynamism
within the factor graph framework. We will evaluate existing approaches of
robust SLAM and conduct experiments on real world datasets to validate the
performance of our approach.
