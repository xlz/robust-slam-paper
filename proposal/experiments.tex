\section{Preliminary Results}

In this section, we evaluate the max-mixture algorithm well-known for its
capability of handling a large amount of incorrect loop closures. Since the
max-mixture algorithm is a representative framework in the robust SLAM
literature, its performance against dynamic environments will have border
implication on later work.

The evaluation is based on the Victoria Park dataset with different crafted
noises introduced on a single landmark (landmark 249). The dataset contains
10607 measurements, 151 landmarks, and 6969 poses. Figure XX shows a local
part of the trajectory and map estimation obtained by the max-mixture
algorithm on three instances of the dataset.

The top plot is the result on the originally dataset with all observations of
landmark 249 converted to unimodal max-mixture distributions. SLAM graph
optimization converges with chi-square of 17287.

In the middle plot, all observations of landmark 249 are displaced with a 10
meter southward bias, and converted to unimodal max-mixture distributions. In
this part, SLAM graph optimization converges with chi-square of 231898,
indicating highly uncertain results. Also, almost half of observations are
rejected by the max-mixture algorithm, while the resulting trajectory only
suffers from slight distortion with landmark 249 correctly displaced by the
amount of the bias in the resulting map.

In th bottom plot, all observations of landmark 249 are added with offsets in
a way to simulate that the landmark is moving slowly southward. The total
movement of landmark 249 is 7 meter. As a result, SLAM graph optimization
converges with a low chi-square at 17807, and the resulting trajectory is
largely distored with all observations of landmark 249 being accepted by the
max-mixture algorithms.

This demonstrates how the max-mixture algorithm is unable to handle a locally
consistent moving landmark while capable of handling noise of large
displacement and spurious loop closures. 
