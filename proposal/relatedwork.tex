\section{Related Work}

One solution to moving objects is combining SLAM with object detection and
tracking. In SLAM-DATMO, the authors developed a Bayesian framework to solve
the SLAM together with detection and tracking of moving objects. They employed
sophisticated object detection and tracking and data association algorithms to
model object motion. This approach is suitable for the kind of data such as
dense pointclouds produced by laser scanners for the purpose of obtaining fast
and accurate estimation of motion. However, in the case of sparse landmarks or
features generated by visual sensors, there is usually less than sufficient
data to achieve the same level of performance in object detection and tracking.
Moreover, object detection and tracking is not a part of the SLAM framework per
se, instead it is used as a preprocessing filter to prevent moving objects from
corrupting the input of the SLAM framework. If the goal is just to estimate the
robot trajectory and the still part of the map for future localization, a large
part of the work maintaining motion models of moving objects will not be
essential to the SLAM problem.

Several methods without relying on prefiltering propose additional latent
variables to represent the dynamism in a particular measurement.  Haehnel03
proposes an additional binary variable for each data item to represent whether
or not the data item is caused by moving objects, and uses EM algorithm to
optimize for the best classification of moving objects while solving the SLAM
estimation. 

Similarly, in Switchable12
\{mni\} To gain robustness against false positive loop closures, the
switchable constraints approach allows the optimizer to be able to naturally
change the topological structure of the problem during the optimization
itself, which significantly increases the robustness against outliers of the
whole SLAM system and closes the gap between the front-end and the back-end.
This way, edges representing outlier constraints can be removed from the graph
during the optimization. This is achieved by augmenting the original problem
and introducing an additional type of hidden variable: A switch variable is
associated with each factor that could potential represent an outlier. This
additional variable acts as a multiplicative scaling factor on the information
matrix associated with that constraint. Depending on the state of the switch
variable (a value between 0 and 1), the resulting information matrix is either
the original matrix (when the switch is equal to 1) or 0 (when the switch is
0) or something between both ends. Notice that if the switch variable is equal
to 0, the associated constraint is completely removed and has no influence on
the overall solution. 

Since in pose graph SLAM, every loop closure factor could be an outlier, we
associate each loop closure edge with one of the newly introduced switch
variables. With the switchable constraints, the optimization therefore works
on an augmented problem, searching for the joint optimal configuration of the
original variables and the newly introduced switch variables, here searching
the optimal graph topology. 

The most similar work to ours is \cite{rogers2010slam}, which uses the same formulations as we did in figure \ref{fig:model}. They added a indicator variable to every landmark and use EM algorithms to iteratively infer those additional variables in the graphical model. However, the EM algorithm is not online -- if there's a new observation we need to run the whole algorithm again. We want to address this issue by designing algorithms that is reasonable in performance while fast enough to compute.


These methods based on EM algorithm suffer from the limitation of EM algorithm
in which they do not have the guarantee of convergence in specified number of
iterations, making them unsuitable for real-time applications.
