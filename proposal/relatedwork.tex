\section{Related Work}
related work

\{mni\} To gain robustness against false positive loop closures, the
switchable constraints approach allows the optimizer to be able to naturally
change the topological structure of the problem during the optimization
itself, which significantly increases the robustness against outliers of the
whole SLAM system and closes the gap between the front-end and the back-end.
This way, edges representing outlier constraints can be removed from the graph
during the optimization. This is achieved by augmenting the original problem
and introducing an additional type of hidden variable: A switch variable is
associated with each factor that could potential represent an outlier. This
additional variable acts as a multiplicative scaling factor on the information
matrix associated with that constraint. Depending on the state of the switch
variable (a value between 0 and 1), the resulting information matrix is either
the original matrix (when the switch is equal to 1) or 0 (when the switch is
0) or something between both ends. Notice that if the switch variable is equal
to 0, the associated constraint is completely removed and has no influence on
the overall solution. 

Since in pose graph SLAM, every loop closure factor could be an outlier, we
associate each loop closure edge with one of the newly introduced switch
variables. With the switchable constraints, the optimization therefore works
on an augmented problem, searching for the joint optimal configuration of the
original variables and the newly introduced switch variables, here searching
the optimal graph topology. 
