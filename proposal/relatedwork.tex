\section{Related Work}

There are multiple recent successful optimization techniques for graph SLAM
based on similar formulations. iSAM\cite{isam} simultaneously acquires optimal
smoothing estimates of the whole trajectory and the map by converting the
graph SLAM maximum likelihood estimate into a non-linear least squares
optimization problem, which is then incrementally solved by numerical methods,
obtaining real-time performance and smoother accuracy. These successful
techniques show the effectiveness of the factor graph formulation of the SLAM
problem, and we base our formulation in similar forms.

A known solution to moving objects in the environment is combining SLAM with
object detection and tracking. In \cite{wang2003online}, the authors developed a
Bayesian framework to solve the SLAM together with detection and tracking of
moving objects. They employed sophisticated object detection and tracking and
data association algorithms to model object motion. This approach is suitable
for the kind of data such as dense point-clouds produced by laser scanners for
the purpose of obtaining fast and accurate estimation of motion. However, in
the case of sparse landmarks or features generated by visual sensors, there is
usually less than sufficient data to achieve the same level of performance in
object detection and tracking.  Moreover, object detection and tracking is not
a part of the SLAM framework per se, instead it is used as a preprocessing
filter to prevent moving objects from corrupting the input of the SLAM
framework. If the goal is just to estimate the robot trajectory and the still
part of the map for future localization, a large part of the work maintaining
motion models of moving objects will not be essential to the SLAM problem. In
our approach, we do not explicitly model the motion of individual objects,
instead we maintain a coherent reference frame in order to obtain only the
trajectory and the map and discard irrelevant or potentially moving landmarks.
This simplifies the pipeline and removes a chunk of computation for increased
efficiency.

In terms of front-end outliers, several robust SLAM approaches have been
proposed so far without relying on pre-filtering. Some use robust objective
functions or robust representation of graph factors, such as
Max-Mixture\cite{mm} which represents factors not as a single spatial
Gaussian distribution, but as a mixture of Gaussians, which can handle
occasional outliers efficiently. The problem with this kind of approaches is
that they mainly considers perceptual aliasing errors stemmed from wrong loop
closures, and they assume landmarks to be immovable in static environments.
Unfortunately such assumptions do not hold for the application of robot
navigation in social environments, where there will be people and objects
moving around. If a whole block of presumed ``landmarks'' change their
locations, Max-Mixture, for example, will not be able to handle the plausible
errors.

%Besides robust methods, several approaches integrate the problem of
%determining mobility into the back-end graph optimization framework.  To gain
%robustness against false positive loop closures, the switchable
%constraints\cite{Switchable12} approach allows the optimizer to be able to
%naturally change the topological structure of the problem during the
%optimization itself, which significantly increases the robustness against
%outliers of the whole SLAM system and closes the gap between the front-end and
%the back-end.  This way, edges representing outlier constraints can be removed
%from the graph during the optimization.  This is achieved by augmenting the
%original problem and introducing an additional type of hidden variable: A
%switch variable is associated with each factor that could potential represent
%an outlier. This additional variable acts as a multiplicative scaling factor
%on the information matrix associated with that constraint. Depending on the
%state of the switch variable, the resulting
%information matrix is either the original matrix or 0
%or something between both ends. With the
%switchable constraints, the optimization therefore works on an augmented
%problem, searching for the joint optimal configuration of the original
%variables and the newly introduced switch variables, here searching the
%optimal graph topology. We use similar model as \cite{haehnel03iros}
%\cite{rogers2010slam} in Figure \ref{fig:model}, with a latent indicator
%variable to each landmark indicating whether it is mobile and using the EM
%algorithm to iteratively infer those additional variables in the graphical
%model and estimate the SLAM problem.

However, these methods based on the EM algorithm suffer from the limitation of
the EM algorithm in which they do not have the guarantee of convergence in
specified number of iterations, making them unsuitable for real-time
applications where the optimization must complete before the next measurement
arrives. Our approach will address this issue by designing incremental
algorithms that is reasonable in performance while fast enough to compute.

